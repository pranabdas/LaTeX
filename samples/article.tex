\documentclass{article}

\usepackage[utf8]{inputenc}
\usepackage[margin=1in]{geometry}
\usepackage[titletoc,title]{appendix}

\usepackage{amsmath,amsfonts,amssymb,mathtools}

\usepackage{graphicx,float}

\usepackage[ruled,vlined]{algorithm2e}
\usepackage{algorithmic}

\usepackage{biblatex}
\addbibresource{references.bib}

\title{Report title}
\author{Pranab Das}
\date{\today}

\begin{document}

\maketitle

% Abstract
\begin{abstract}
    Add your abstract here.
\end{abstract}

% Introduction and Overview
\section{Introduction and Overview}
Add your introduction and overview here.

% Example Subsection
\subsection{Subsection Title}
This is a subsection.

% Example Subsubsection
\subsubsection{Subsubsection Title}
This is a subsubsection.

%  Theoretical Background
\section{Theoretical Background}
Add your theoretical background here. Some example text: As we learned from our textbook \cite{kutz_2013}, Fourier introduced the concept of representing a given function $f(x)$ by a trigonometric series of sines and cosines:
\begin{equation}
    f(x) = \frac{a_0}{2} + \sum_{i=1}^\infty \left(a_n\cos{nx} + b_n\sin{nx}\right) \quad x \in (-\pi,\pi].
    \label{eqn:fourierseries}
\end{equation}
You can reference numbered equations, figures, tables, algorithms, and code like this: Equation~\ref{eqn:fourierseries}, etc.

% Algorithm Implementation and Development
\section{Algorithm Implementation and Development}
Add your algorithm implementation and development here. See Algorithm~\ref{alg:example} for how to include an algorithm in your document. This is how to make an \textit{ordered} list:
\begin{enumerate}
    \item Fluffy swallowed a marble.
    \item I took Fluffy to the vet.
    \item They took an ultrasound of Fluffy's intestines.
\end{enumerate}


% Computational Results
\section{Computational Results}
Add your computational results here. See Table~\ref{tab:mascots} for how to include a table in your document. See Figure~\ref{fig:dubs} for how to include figures in your document.

\begin{table}
    \centering
    \begin{tabular}{rll}
    & Name & Years \\
    \hline
    1 & Frosty & 1922-1930  \\
    2 & Frosty II & 1930-1936 \\
    3 & Wasky & 1946 \\
    4 & Wasky II & 1947 \\
    5 & Ski & 1954 \\
    6 & Denali & 1958 \\
    7 & King Chinook & 1959-1968\\
    8 & Regent Denali & 1969 \\
    9 & Sundodger Denali & 1981-1992 \\
    10 & King Redoubt & 1992-1998 \\
    11 & Prince Redoubt & 1998 \\
    12 & Spirit & 1999-2008 \\
    13 & Dubs I & 2009-2018 \\
    14 & Dubs II & 2018-Present
    \end{tabular}
    \caption{UW mascots as described in \cite{washington_huskies}.}
    \label{tab:mascots}
\end{table}

% Summary and Conclusions
\section{Summary and Conclusions}
Add your summary and conclusions here.

% References
\printbibliography

% Appendices
\begin{appendices}

% MATLAB Functions
\section{MATLAB Functions}
Add your important MATLAB functions here with a brief implementation explanation. This is how to make an \textbf{unordered} list:
\begin{itemize}
    \item \texttt{y = linspace(x1,x2,n)} returns a row vector of \texttt{n} evenly spaced points between \texttt{x1} and \texttt{x2}. 
    \item \texttt{[X,Y] = meshgrid(x,y)} returns 2-D grid coordinates based on the coordinates contained in the vectors \texttt{x} and \texttt{y}. \text{X} is a matrix where each row is a copy of \texttt{x}, and \texttt{Y} is a matrix where each column is a copy of \texttt{y}. The grid represented by the coordinates \texttt{X} and \texttt{Y} has \texttt{length(y)} rows and \texttt{length(x)} columns.  
\end{itemize}

% MATLAB Codes
\section{MATLAB Code}
Add your MATLAB code here. This section will not be included in your page limit of six pages.


\end{appendices}

\end{document}
